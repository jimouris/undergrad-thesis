\documentclass{dithesis}

\usepackage{mathspec}
\usepackage{pgfplots}
\usepackage{tikz}
\usepackage{booktabs}
\usepackage{listings}
\usepackage{xcolor}
\usepackage{caption}
\usepackage{fontenc}
\usepackage{xgreek}
\usepackage[english, greek]{babel}
\usepackage[utf8x]{inputenc}
\usepackage{xunicode}

\setmainfont{Times New Roman}
% \setallmainfonts[Mapping=tex-text]{Arial}
% \setallsansfonts[Mapping=tex-text]{Arial}
% \setallmonofonts[Mapping=tex-text]{Arial}

\renewcommand{\university}{National and Kapodistrian University of Athens}
\renewcommand{\school}{School of Science}
\renewcommand{\department}{Department of Informatics and Telecommunications}
\renewcommand{\thesisplace}{Athens}
\renewcommand{\thesisdate}{July 2016}
\renewcommand{\thesislabel}{Undergraduate Thesis}
\renewcommand{\supervisorlabel}{Supervisors}
\renewcommand{\idlabel}{R.N.}


\begin{document}

\begin{otherlanguage}{english}
\thesistitle{Doop-Soot: Parallel Fact Generation}
\thesisauthor{Dimitris I. Mouris}{1115201200114}
\supervisor{Yannis Smaragdakis}{Associate Professor NKUA}
\supervisor{Anastasis Antoniadis}{M.Sc. Student NKUA}
\maketitle
\setcounter{page}{3}
\end{otherlanguage}


\begin{otherlanguage}{greek}
\renewcommand{\university}{Εθνικό και Καποδιστριακό Πανεπιστήμιο Αθηνών}
\renewcommand{\school}{Σχολή Θετικών Επιστημών}
\renewcommand{\department}{Τμήμα Πληροφορικής και Τηλεπικοινωνιών}
\renewcommand{\thesisplace}{Αθήνα}
\renewcommand{\thesisdate}{Ιούλιος 2016}
\renewcommand{\thesislabel}{Πτυχιακή Εργασία}
\renewcommand{\supervisorlabel}{Επιβλέποντες}
\renewcommand{\idlabel}{Α.Μ.}
\thesistitle{Doop-Soot: Παραλληλοποίηση της Δημιουργίας Γεγονότων}
\thesisauthor{Δημήτρης Η. Μούρης}{1115201200114}
\let\thesupervisorslist\undefined
\supervisor{Γιάννης Σμαραγδάκης}{Αναπληρωτής Καθηγητής ΕΚΠΑ}
\supervisor{Αναστάσης Αντωνιάδης}{Μεταπτυχιακός Φοιτητής ΕΚΠΑ}
\maketitlesecond
\end{otherlanguage}

\thesistitle{Doop-Soot: Parallel Fact Generation}
\thesisauthor{Dimitris I. Mouris}{1115201200114}

\begin{otherlanguage}{greek}
\begin{thesisabstract}[ΠΕΡΙΛΗΨΗ]
    Παραλληλοποίηση του Fact Generation του Doop. Το Doop χρησιμοποιείται για μπλαμαπλπαλαμπλ anastasi grapse kati edw

    \thesiskeywords{Θεματική Περιοχή}{Τεκμηρίωση}
                {Λέξεις Κλειδιά}{Static Program Analysis}
                                {Doop: Fact Generation}
                                {Soot}
                                {Πτυχιακές Εργασίες}
                                {Τμήμα Πληροφορικής και Τηλεπικοινωνιών}
                                {Πανεπιστήμιο Αθηνών}
\end{thesisabstract}
\end{otherlanguage}

\begin{thesisabstract}[ABSTRACT]
    Parallelizing doop fact generation blablahvblah

    \thesiskeywords{Subject Area}{Documentation}
                {Keywords}{Static Program Analysis}
                            {Doop: Fact Generation}
                            {Soot}
                            {Undergraduate Thesis}
                            {Dept. of Informatics}
                            {University of Athens}
\end{thesisabstract}

\begin{thesisdedication}
to someone.
\end{thesisdedication}

\begin{thesisacknowledgments}[Acknowledgements]
    I would like to thank my supervisor, Prof. Yannis Smaragdakis for the chance he gave me with this project and also for his support and help all these months.
    \\
    \\
    My sincere thanks also goes to M.Sc. candidate and my friend Anastasis Antoniadis for his patience and guidance which had been very helpfull to finish this work.
\end{thesisacknowledgments}

\renewcommand\contentsname{Contents}
\renewcommand\listfigurename{List of Figures}
\renewcommand\listtablename{List of Tables}
\renewcommand{\figurename}{Figure}
\renewcommand{\cftfigpresnum}{Figure }
\renewcommand{\tablename}{Table}
\renewcommand{\cfttabpresnum}{Table }


\tableofcontents
% \addcontentsline{toc}{section}{LIST OF FIGURES}
\listoffigures
% \addcontentsline{toc}{section}{LIST OF TABLES}
\listoftables

\definecolor{gray}{rgb}{0.5,0.5,0.5}
\definecolor{mmaroon}{HTML}{800000}
\definecolor{rred}{HTML}{C0504D}
\definecolor{oorange}{HTML}{FFA500}
\definecolor{ggreen}{HTML}{9BBB59}
\definecolor{bblue}{HTML}{4F81BD}
\definecolor{ccyan}{HTML}{00FFFF}
\lstset{frame=tb, language=Java, aboveskip=3mm, belowskip=3mm,
    showstringspaces=false, columns=flexible,
    basicstyle={\small\ttfamily}, numbers=left, framexleftmargin=20pt, commentstyle=\color{gray},
    numberstyle=\tiny\color{gray}, breaklines=true, breakatwhitespace=true, tabsize=3
}


\begin{thesisprologue}[Preface]
    This project was developed in Athens, Greece between September 2015 and March 2016. At the very beginning of this work, it was essential to understand how the Doop framework works for pointer or points-to analysis. Equally important was the task of understanding the codebase of the Soot framework and growing familiar with its integration with Doop in order to translate Java bytecode to an IR (Jimple) and produce the facts for the analysis. The core of this work was focused on understanding the way Java bytecode to Jimple translation is implemented and attempting to parallelize it without disrupting the Doop workflow.
\end{thesisprologue}

\thesissection{Introduction}
Soot is a Java bytecode optimization framework which my colleagues use for fact generation in order to perform points-to analysis of Java programs, in Datalog, using the Doop framework. For this task, I had to parallelize the fact generation process and proceed to the appropriate modifications in both Soot and Doop.

This thesis aims ...

The rest of the thesis is organized as follows:
...
\thesissection{Doop}
    Doop is a framework for pointer, or points-to, analysis of Java programs. Doop implements a range of algorithms, including context insensitive, call-site sensitive, and object-sensitive analyses, all specified modularly as variations on a common code base.

	From the Doop website: \\
    \textit{"Doop builds on the idea of specifying pointer analysis algorithms declaratively, using Datalog: a logic-based language for defining (recursive) relations. Doop carries the declarative approach further than past work by describing the full end-to-end analysis in Datalog and optimizing aggressively through exposition of the representation of relations (for example indexing) to the Datalog language level. Doop uses the Datalog dialect and engine of LogicBlox."}

    Compared to alternative context-sensitive pointer analysis implementations (such as Paddle) Doop is much faster, and scales better. Also, with comparable context-sensitivity features, Doop is more precise in handling some Java features (for example exceptions) than alternatives. //kati prepei na alla3ei edw..

    Doop is launched by specifying the type of analysis to run and the target directory that contains java bytecode (.jar file). At first, the facts are generated by soot and imported to a database -or more precisely a knowlegdebase- and then the analysis the specified is run. \cite{Doop: Framework for Java Pointer Analysis}

    \thesissubsection{Fact Generation}
        Doop before running a pointer or points-to analysis, intergrates with Soot to generate either Jimple (\textbf{J}ava s\textbf{imple}) or Shimple (an \textbf{S}sa version of J\textbf{imple}) intermidiate representations. Jimple is a typed 3-address IR suitable for performing optimizations; it only has 15 statements. Then from jimple the facts are generated and imported into a database with multiple tables. Shimple is an SSA-version of jimple; obviously first jimple is generated and then Soot applies a transformation pack to jimple body to create shimple.

        The main motivation behind this thesis was the fact that the sequential fact generation time amounts to more than 50\% of the total execution time (either for jimple or shimple). Below are presented a few time examples of the sequential Doop Fact Generation and total execution time (both fact generation and analysis times).

    \thesissubsection{Doop Time Examples}
        \begin{table}[H]
            \centering
                \begin{tabular}{llll}
                \hline
                \textbf{Soot 2.5.0}      & \textbf{antlr.jar} & \textbf{hsqldb.jar} & \textbf{batik.jar} \\ \hline
                \textbf{Fact Generation} & 1.16 min.          & 1.23 min.           & 2.26 min.          \\
                \textbf{Total time}      & 3.18 min.          & 3.21 min.           & 4.34 min.          \\ \hline
                \end{tabular}
                \newline
            \caption{Soot 2.5.0 times}
        \end{table}

    \thesissubsection{Fact Table Example}
    	Here are presented some of the facts that created for the \texttt{helloWorld} example.
    	\begin{figure}[H]
\begin{lstlisting}
0   java.io.PrintStream.requireNonNull/java.lang.NullPointerException.<init>/0  java.io.PrintStream.requireNonNull/r1
0   helloWorld.main/java.io.PrintStream.println/0   helloWorld.main/$stringconstant0
0   java.io.PrintStream.toCharset/java.io.PrintStream.requireNonNull/0  java.io.PrintStream.toCharset/r0
1   java.io.PrintStream.toCharset/java.io.PrintStream.requireNonNull/0  java.io.PrintStream.toCharset/$stringconstant0
/* ... */
\end{lstlisting}
\caption{ActualParam.facts}
        \end{figure}



        \begin{figure}[H]
\begin{lstlisting}
helloWorld.<init>/definition/instruction1   1   helloWorld.<init>/@this helloWorld.<init>/r0    <helloWorld: void <init>()>
helloWorld.main/definition/instruction1 1   helloWorld.main/@param0 helloWorld.main/r0  <helloWorld: void main(java.lang.String[])>
/* ... */
\end{lstlisting}
\caption{AssignLocal.facts}
        \end{figure}



        \begin{figure}[H]
\begin{lstlisting}
<class helloWorld>      java.lang.Class helloWorld
<class java.io.PrintStream>     java.lang.Class java.io.PrintStream
<class java.lang.Object>        java.lang.Class java.lang.Object
<class java.lang.String>        java.lang.Class java.lang.String
<class java.lang.System>        java.lang.Class java.lang.System
/* ... */
\end{lstlisting}
\caption{ClassObject.facts}
        \end{figure}



        \begin{figure}[H]
\begin{lstlisting}
java.io.PrintStream.requireNonNull/assign/instruction4  4   java.io.PrintStream.requireNonNull/new java.lang.NullPointerException/0 java.io.PrintStream.requireNonNull/$r2  <java.io.PrintStream: java.lang.Object requireNonNull(java.lang.Object,java.lang.String)>
helloWorld.main/invoke/instruction3 3   helloWorld  helloWorld.main/$stringconstant0    <helloWorld: void main(java.lang.String[])>
java.io.PrintStream.toCharset/invoke/instruction2   2   charsetName java.io.PrintStream.toCharset/$stringconstant0  <java.io.PrintStream: java.nio.charset.Charset toCharset(java.lang.String)>
/* ... */
\end{lstlisting}
\caption{AssignHeapAllocation.facts}
        \end{figure}



        \begin{figure}[H]
\begin{lstlisting}
helloWorld
java.lang.Class
java.io.PrintStream
java.lang.Class
/* ... */
\end{lstlisting}
\caption{ClassType.facts}
        \end{figure}


\thesissection{Soot}
    Originally, Soot started off as a Java optimization framework. By now, researchers and practitioners from around the world use Soot to analyze, instrument, optimize and visualize Java and Android applications. Soot provides four intermediate representations for analyzing and transforming Java bytecode:
	\begin{enumerate}
		\item Baf: a streamlined representation of bytecode which is simple to manipulate.
		\item Jimple: a typed 3-address intermediate representation suitable for optimization.
		\item Shimple: an SSA variation of Jimple.
		\item Grimp: an aggregated version of Jimple suitable for decompilation and code inspection.
		\item Jimple is Soot’s primary IR and most analyses are implemented on the Jimple level. Custom IRs may be added when desired.
	\end{enumerate}
    In our case, we use Soot to generate Jimple (or Shimple) facts from Java bytecode to run a pointer or points-to analysis.
    \cite{Sable: Soot}

    \thesissubsection{Applying the latest Soot version}
    	To minimize the time consumed in fact generation phase, before we tried to parallelize it, we applied the latest (Sept. 2015) Soot version. We faced a lot of compatibility problems in order to use it properly but it gave a speed up of 40\%. Below are presented a few time examples with Soot-2.5.0 and the latest one.

		\begin{figure}[H]
			\centering
\begin{tikzpicture}
\begin{axis}[ width=\textwidth, height=12cm, major x tick style = transparent, ybar=2*\pgflinewidth, bar 		width=14pt, ymajorgrids = true, ylabel = {time (sec)}, symbolic x coords={antlr, hsqldb, batik}, xtick = data, scaled y ticks = false, enlarge x limits=0.25, ymin=0, legend cell align=left, legend style={at={(1,1.05)}, anchor=south east, column sep=1ex} ]
\addplot[style={rred,fill=rred,mark=none}]
	coordinates {(antlr, 76) (hsqldb, 83) (batik, 146)};
\addplot[style={bblue,fill=bblue,mark=none}]
    coordinates {(antlr, 48) (hsqldb, 50) (batik, 63)};
\legend{Soot 2.5.0, Soot Latest Version}
\end{axis}
\end{tikzpicture}
			\caption{Fact Generation with different Soot versions}
		\end{figure}

    	Then, to gain more speedup we tried to parallelize the fact generation part, which spends a similar amount of a simple analysis time. In order to do that, we had to understand the way bytecode is translated to jimple. Below we explain in more detail this procedure.

    \thesissubsection{Bytecode To Jimple}
        As we mentioned before, Soot is able to translate Java bytecode to a typed 3-address IR, Jimple. Jimple (\textbf{J}ava s\textbf{imple}) is a very convinient IR for performing optimizations, it only has 15 statements.

        Soot has various phases and a lot of different options for transformations given. The one that is responsible for bytecode to jimple translation is jb phase. In this phase, first Soot translates bytecode to untyped Jimple and introduces new local variables; Jimple is stackless, Soot is using variables for stack locations. Then it inferres Types to the untyped jimple. The next step is to linearize all the expressions to statements that only reference at most 3 local variables or constants.

        Getting a little deeper, in a general case the way Soot handles Java bytecode classes is the following: \\
        Soot is launched by specifying a directory with the Application code as a parameter. The way Doop intergrates with Soot, it is only Java bytecode, either a class file or a jar. First, the \texttt{main()} method of the Main class is executed and calls \texttt{Scene.loadNecessaryClasses()} (In our case Doop calls \texttt{Scene.loadNecessaryClasses()} not Main class). This method loads basic Java classes and then loads specific Application classes by calling \texttt{loadClass()}. Then, \texttt{SootResolver.resolveClass()} is called. The resolver calls \texttt{SourceLocator.getClassSource()} to fetch a reference to a ClassSource, an interface between the file containing the Java bytecode and Soot. For Java bytecode to Jimple translation the class source is a \texttt{CoffiClassSource} because it is the coffi module which handles this conversion. Then, the resolver having a reference to a class source, calls \texttt{resolve()} on it. This methods in turn calls \texttt{soot.coffi.Util.resolveFromClassFile()} which creates a SootClass from the corresponding Java bytecode class. All source fields of Soot class methods are set to refer to a CoffiMethodSource object. This object is used later to get the Jimple representation of the method. For example, if during an analysis with Soot the analysis code calls \texttt{SootMethod.getActiveBody()} and the Jimple code of the method was not already generated, \texttt{getActiveBody()} will call \texttt{CofficMethodSource.getBody()} to compute Jimple code from the Java bytecode. Actually this method \texttt{getActiveBody()} spends the most of the Java bytecode to Jimple conversion time. The Jimple code representation of the method can then be analyzed and/or transformed.

    \thesissubsection{Compiling \& Running Soot}
    	Soot uses ant \cite{Apache Ant} for compiling and building the project. With minor modifications in the \texttt{ant.settings} file, it is very easy to compile and run every version of Soot.
        \begin{figure}[H]
\begin{lstlisting}
ant                 /* To compile */
ant classesjar      /* To generate the sootclasses jar file */
ant fulljar         /* To generate the complete soot jar file */
\end{lstlisting}
        \caption{Compiling Soot}
        \end{figure}

        Running Soot and generating Jimple from java bytecode or from a jar file is presented below. (For Shimple IR, -ssa flag is needed).
        \begin{figure}[H]
\begin{lstlisting}
(create a test.java)
javac test.java
java -cp ./lib/soot-trunk.jar soot.Main -f J -cp .:/usr/lib/jvm/java-7-openjdk-amd64/jre/lib/rt.jar test
\end{lstlisting}
        \caption{Generating Jimple from .class}
        \end{figure}

        \begin{figure}[H]
\begin{lstlisting}
java -cp ./lib/soot-trunk.jar soot.Main -f J -cp .:/usr/lib/jvm/java-7-openjdk-amd64/jre/lib/rt.jar -process-dir pathtotest.jar
\end{lstlisting}
        \caption{Generating Jimple from .jar}
        \end{figure}

    \thesissubsection{Jimple Examples}
        Below are two simple java programs along with their jimple translation. The first one is the classic HelloWorld, and the second is a simple inheritance test that depends on the user's input.
       	The local variables which start with a \$ sign represent stack positions and not local variables in the original program whereas those without \$ represent real local variables.
        \thesissubsubsection{Hello World}
            \begin{figure}[H]
\begin{lstlisting}
public class helloWorld {
   public static void main(String[] args) {
       System.out.println("Hello, World");
   }
}
\end{lstlisting}
            \caption{HelloWorld.java}
            \end{figure}
            \begin{figure}[H]
\begin{lstlisting}
public class helloWorld extends java.lang.Object {

    public void <init>() {
        helloWorld r0;
        r0 := @this: helloWorld;
        specialinvoke r0.<java.lang.Object: void <init>()>();
        return;
    }

    public static void main(java.lang.String[]) {
        java.lang.String[] r0;
        java.io.PrintStream $r1;
        r0 := @parameter0: java.lang.String[];
        $r1 = <java.lang.System: java.io.PrintStream out>;
        virtualinvoke $r1.<java.io.PrintStream: void println(java.lang.String)>("Hello, World");
        return;
    }

}
\end{lstlisting}
            \caption{HelloWorld.jimple}
            \end{figure}

        \thesissubsubsection{Inheritance Test}
            \begin{figure}[H]
\begin{lstlisting}
public class inheritanceTest {
    public static void main(String[] args) {
        testA a;
        if (args.length < 1) {
            a = new testA(5);
        } else {
            a = new testB(5);
        }
        int result = a.getA();
        System.out.println("the value of a is " + result);
    }

    public static class testA {
        int a;

        public testA(int a) {
            this.a = a;
        }

        public int getA() {
            return this.a;
        }
    }

    public static class testB extends testA {
        public testB(int a) {
            super(a+100);
        }
    }
}
\end{lstlisting}
            \caption{inheritanceTest.java}
            \end{figure}
            \begin{figure}[H]
\begin{lstlisting}
public class inheritanceTest extends java.lang.Object {
    public void <init>() {
        inheritanceTest r0;
        r0 := @this: inheritanceTest;
        specialinvoke r0.<java.lang.Object: void <init>()>();
        return;
    }

    public static void main(java.lang.String[]) {
        java.lang.String[] r0;
        int $i0, i1;
        inheritanceTest$testA $r1, r2;
        inheritanceTest$testB $r3;
        java.io.PrintStream $r4;
        java.lang.StringBuilder $r5, $r6, $r7;
        java.lang.String $r8;
        r0 := @parameter0: java.lang.String[];
        $i0 = lengthof r0;
        if $i0 >= 1 goto label1;
        $r1 = new inheritanceTest$testA;
        specialinvoke $r1.<inheritanceTest$testA: void <init>(int)>(5);
        r2 = $r1;
        goto label2;
     label1:
        $r3 = new inheritanceTest$testB;
        specialinvoke $r3.<inheritanceTest$testB: void <init>(int)>(5);
        r2 = $r3;
     label2:
        i1 = virtualinvoke r2.<inheritanceTest$testA: int getA()>();
        $r4 = <java.lang.System: java.io.PrintStream out>;
        $r5 = new java.lang.StringBuilder;
        specialinvoke $r5.<java.lang.StringBuilder: void <init>()>();
        $r6 = virtualinvoke $r5.<java.lang.StringBuilder: java.lang.StringBuilder append(java.lang.String)>("the value of a is ");
        $r7 = virtualinvoke $r6.<java.lang.StringBuilder: java.lang.StringBuilder append(int)>(i1);
        $r8 = virtualinvoke $r7.<java.lang.StringBuilder: java.lang.String toString()>();
        virtualinvoke $r4.<java.io.PrintStream: void println(java.lang.String)>($r8);
        return;
    }
}
\end{lstlisting}
            \caption{inheritanceTest.jimple}
            \end{figure}



\thesissection{Parallelizing Fact Generation}
    We now describe the basic idea of Fact Generation from the Doop side. Given all the classes (sootClasses) to generate, Doop iterates each one of them; writes all the superClasses, if exist, and then generate all fields (sootFields) and methods (sootMethods). Below is presented the \texttt{FactGenerator.java} which implements the work described above and then calls Soot.
    \begin{figure}[H]
\begin{lstlisting}
public class FactGenerator {
    /* ... */

    public void generate(sootClass) {
        if(c.hasSuperclass() && !c.isInterface())
            _writer.writeDirectSuperclass(c, c.getSuperclass());
        for(SootField f : c.getFields())
            generate(f);
        for(SootMethod m : c.getMethods()) {
            Session session = new Session();
            generate(m, session);
        }
    }

    public void generate(SootMethod m, Session session) {
        /* ... */

        /* This instruction spends more than 80% of FG time */
        m.retrieveActiveBody()

        /* ... */
    }

    /* ... */
}
\end{lstlisting}
    \caption{Sequential Fact Generation}
    \end{figure}

    Having the previous basic structure in mind, and considering that \texttt{m.retrieveActiveBody()} spends more than 80\% of total fact generation time, we tried to parallelize the method which calls \texttt{m.retrieveActiveBody()}. In order to do that, we approached the problem in four ways. The three of them are pretty much similar while the other one is based on a recursive Java Framework (Fork/Join Framework). Below are presented some Fact Generation time examples with the sequential FG.
    \begin{table}[H]
		\centering
        \begin{tabular}{@{}ll@{}}
        \toprule
        \textbf{Jars} & \textbf{Time (sec.)} \\ \midrule
        antlr          & 48                    \\
        eclipse        & 27                    \\
        jython         & 32                    \\
        hsqldb         & 50                    \\
        batik          & 63                    \\ \bottomrule
        \end{tabular}
        \newline
		\caption{Sequential Fact Generation Time Examples}
	\end{table}


    \thesissubsection{One Thread Per Method}
        Our first approach to parallelize Fact Generation is similar as the sequential one, but instead of having a loop over all Soot Methods and call \texttt{generate(m, session)}, we assign the task to a thread for each one of them. We created a new Java class, MethodGenerator, which is identical to FactGenerator and in addition has a \texttt{run()} method to generate sootMethods.
        \begin{figure}[H]
\begin{lstlisting}
public class FactGenerator {
    private ExecutorService MgExecutor = new ThreadPoolExecutor(8, 16, 0L, TimeUnit.MILLISECONDS, new LinkedBlockingQueue<Runnable>());
    /* ... */

    public void generate(sootClass) {
        if(c.hasSuperclass() && !c.isInterface())
            _writer.writeDirectSuperclass(c, c.getSuperclass());
        for(SootField f : c.getFields())
            generate(f);
        for(SootMethod m : c.getMethods()) {
            Session session = new Session();
            Runnable mg = new MethodGenerator();
            MgExecutor.execute(mg);
        }
    }
}

public class MethodGenerator {
    public void run() {
        generate(this.m, this.s)
    }

    /* ... */
}
\end{lstlisting}
        \caption{One Thread Per Method}
        \end{figure}

        Below are presented some Fact Generation time examples with the \texttt{One Thread Per Method FG} approach for various thread-pool sizes (such as 4, 16 and 32).
		\begin{table}[H]
			\centering
            \begin{tabular}{@{}l|lll@{}}
            \toprule
            \textbf{Jars}    	& \multicolumn{3}{l}{\textbf{Time (sec.)}}  \\ \midrule
            \textbf{Pool Size} 	& \textbf{4}  & \textbf{16}  & \textbf{32}  \\ \midrule
            antlr            	& 21          & 14           & 13           \\
            eclipse          	& 13          & 7            & 8            \\
            jython           	& 14          & 9            & 9            \\
            hsqldb           	& 23          & 15           & 16           \\
            batik            	& 26          & 23           & 18           \\ \bottomrule
            \end{tabular}
            \newline
			\caption[One Thread Per Method Time Examples]{One Thread Per Method Time Examples, with pool size: 4, 16, 32}
		\end{table}



    \thesissubsection{One Thread Per Class}
        In our second approach, we tried to find out ways to gain more speedup. So, we observed that some threads did not have much work to do and finishing their task instantly. Allocating a new object and assigning to it a task just to finish instantly was an overhead. As a result, we tried to feed the threads more than just a method, so we created a new thread for each class not for each method.
        \begin{figure}[H]
\begin{lstlisting}
public class FactGenerator {
    private ExecutorService CgExecutor = new ThreadPoolExecutor(8, 16, 0L, TimeUnit.MILLISECONDS, new LinkedBlockingQueue<Runnable>());
    /* ... */

    public void generate(sootClass) {
        Runnable cg = new ClassGenerator();
        CgExecutor.execute(cg);
    }
}

public class ClassGenerator {
    public void run() {
        if(c.hasSuperclass() && !c.isInterface())
            _writer.writeDirectSuperclass(c, c.getSuperclass());
        for(SootField f : c.getFields())
            generate(f);
        for(SootMethod m : c.getMethods()) {
            Session session = new Session();
            Runnable mg = new MethodGenerator();
            MgExecutor.execute(mg);
            generate(m, session);
        }
    }

    /* ... */
}
\end{lstlisting}
        \caption{One Thread Per Class}
        \end{figure}

        The results were slightly better than the previous but without achiving a remarkable speedup. Below are presented some Fact Generation time examples with the \texttt{One Thread Per Class FG} approach for various thread-pool sizes (such as 4, 16 and 32).
        \begin{table}[H]
			\centering
            \begin{tabular}{@{}l|lll@{}}
            \toprule
            \textbf{Jars}    	& \multicolumn{3}{l}{\textbf{Time (sec.)}}  \\ \midrule
            \textbf{Pool Size} 	& \textbf{4}  & \textbf{16}  & \textbf{32}  \\ \midrule
            antlr            	& 21          & 14           & 13           \\
            eclipse          	& 13          & 7            & 8            \\
            jython           	& 14          & 9            & 9            \\
            hsqldb           	& 23          & 15           & 16           \\
            batik            	& 26          & 23           & 18           \\ \bottomrule
            \end{tabular}
            \newline
			\caption[One Thread Per Class Time Examples]{One Thread Per Class Time Examples, with pool size: 4, 16, 32}
		\end{table}


    \thesissubsection{Fork-Join Framework}
        So in order to achive more speedup we tried a completely different approach than the two previous, a recursive Java FrameWork (Fork/Join framework). As Oracle describes in Java Documentation, Fork/Join Framework is
    	\textit{"an implementation of the ExecutorService interface that helps you take advantage of multiple processors. It is designed for work that can be broken into smaller pieces recursively. The goal is to use all the available processing power to enhance the performance of your application.}

		\textit{The center of the fork/join framework is the ForkJoinPool class, an extension of the \\ AbstractExecutorService class. ForkJoinPool implements the core work-stealing algorithm and can execute ForkJoinTask processes.}

    	\textit{The idea of using the fork/join framework is to write code that performs a segment of the work. The basic structure should be like the following pseudocode."} \cite{Oracle Java Fork/Join Framework}

        \begin{figure}[H]
\begin{lstlisting}
if (my portion of the work is small enough) {
    do the work directly
} else {
    split my work into two pieces
    invoke the two pieces and wait for the results
}
\end{lstlisting}
        \caption{Fork-Join Basic-Use}
        \end{figure}

        \begin{figure}[H]
\begin{lstlisting}
public class FactGenerator {
    private ForkJoinPool classGeneratorPool = new ForkJoinPool();
    /* ... */
    public void generate(sootClass) {
        if(c.hasSuperclass() && !c.isInterface())
            _writer.writeDirectSuperclass(c, c.getSuperclass());
        for(SootClass i : c.getInterfaces())
            _writer.writeDirectSuperinterface(c, i);
        for(SootField f : c.getFields())
            generate(f);
        if (c.getMethods().size() > 0) {
            ClassGenerator classGenerator = new ClassGenerator(_writer, _ssa, c, 0, c.getMethods().size());
            classGeneratorPool.invoke(classGenerator);
        }
    }
}

public class ClassGenerator {
    /* ... */
    public void compute() {
        List<SootMethod> sootMethods = _sootClass.getMethods();
        /* if (my portion of the work is small enough) */
        if (_to - _from < threshold) {  /* How many classes can I process? */
            for (int i = _from ; i < _to ; i++) {
                SootMethod m = sootMethods.get(i);
                Session session = new Session();
                generate(m, session);
            }
        } else { /* split work*/
            int half = (_to - _from)/2;
            ClassGenerator c1 = new ClassGenerator(_writer, _ssa, _sootClass, _from, _from + half);
            ClassGenerator c2 = new ClassGenerator(_writer, _ssa, _sootClass, _from + half, _to);
            invokeAll(c1, c2);
        }
    }
    /* ... */
}
\end{lstlisting}
        \caption{Fork-Join Framework}
        \end{figure}

        The results were worse than the two previous approaches (still better than then sequential approach). Below are presented some Fact Generation time examples with the \textit{Fork/Join Framework FG} approach for threshold values (such as 2, 3 and 4). Threshold actually is the number of classes for a thread to process.
		\begin{table}[H]
			\centering
            \begin{tabular}{@{}l|lll@{}}
            \toprule
            \textbf{Jars}    	& \multicolumn{3}{l}{\textbf{Time (sec.)}}  \\ \midrule
            \textbf{Threshold (classes to generate)} 	& \textbf{2}  & \textbf{3}  & \textbf{4}  \\ \midrule
            antlr            	& 23          & 25           & 25           \\
            eclipse          	& 13          & 15           & 18           \\
            jython           	& 16          & 17           & 19           \\
            hsqldb           	& 25          & 28           & 31           \\
            batik            	& 34          & 37           & 38           \\ \bottomrule
            \end{tabular}
            \newline
			\caption[Fork/Join Time Examples]{Fork/Join Time Examples, with threshold 2, 3, 4 and pool size 16}
		\end{table}


    \thesissubsection{Multiple Classes Per Thread}
        Our last approach is similar as the second one, but instead of having one thread per class, we now have one thread per multiple classes. Even in the second approach some threads did not have much work to do. Below is presented in an abstract way the final stage of the code implementing the \textit{Multiple Classes Per Thread} approach.

        \begin{figure}[H]
\begin{lstlisting}
public class FactGenerator {
    public FactGenerator(FactWriter writer, boolean ssa, int totalClasses) {
        _writer = writer;
        _ssa = ssa;
        _classCounter = 0;
        _sootClassArray = new ArrayList<>();
        _totalClasses = totalClasses;
        _cores = Runtime.getRuntime().availableProcessors();
        _executor = new ThreadPoolExecutor(_cores/2, _cores, 0L, TimeUnit.MILLISECONDS, new LinkedBlockingQueue<Runnable>());
    }

    public void generate(sootClass) {
        _classCounter++;
        _sootClasses.add(_sootClass);
        if ((_classCounter % _classSplit == 0) || (_classCounter + _classSplit-1 >= _totalClasses)) {
            Runnable classGenerator = new ClassGenerator(_writer, _ssa, _sootClassArray);
            _classGeneratorExecutor.execute(classGenerator);
            _sootClassArray = new ArrayList<>();
        }
    }
}
\end{lstlisting}
        \caption{Multiple Classes Per Thread: FactGenerator.java}
        \end{figure}

        \begin{figure}[H]
\begin{lstlisting}
public class ClassGenerator {
    public void run() {
        if(c.hasSuperclass() && !c.isInterface())
            _writer.writeDirectSuperclass(c, c.getSuperclass());
        for(SootField f : c.getFields())
            generate(f);
        for(SootMethod m : c.getMethods()) {
            Session session = new Session();
            Runnable mg = new MethodGenerator();
            MgExecutor.execute(mg);
            generate(m, session);
        }
    }

    /* ... */
}
\end{lstlisting}
        \caption{Multiple Classes Per Thread: ClassGenerator.java}
        \end{figure}

        Getting away from the overhead produced by assignments and allocations of the first and second approach, this one had so far the best time results. Below are presented some Fact Generation time examples with the \textit{Multiple Classes Per Thread FG} approach for various thread-pool sizes (such as 4, 16 and 32) and various number of classes per thread.
\begin{table}[H]
        \centering
        \begin{tabular}{@{}lc|lll@{}}
        \toprule
        \textbf{}     & \multicolumn{1}{l|}{\textbf{}}                                                      & \multicolumn{3}{l}{\textbf{Time (sec.)}}                                                                                                                          \\
        \textbf{Jars} & \textbf{\begin{tabular}[c]{@{}c@{}}Classes Per Thread \\ Pool Size\end{tabular}} & \textbf{2}                                           & \textbf{3}                                           & \textbf{4}                                           \\ \midrule
        antlr         & \textbf{\begin{tabular}[c]{@{}c@{}}4\\ 16\\ 32\end{tabular}}                        & \begin{tabular}[c]{@{}l@{}}22\\ 13\\ 14\end{tabular} & \begin{tabular}[c]{@{}l@{}}18\\ 12\\ 13\end{tabular} & \begin{tabular}[c]{@{}l@{}}19\\ 14\\ 13\end{tabular} \\
                      & \multicolumn{1}{l|}{}                                                               &                                                      &                                                      &                                                      \\
        eclipse       & \textbf{\begin{tabular}[c]{@{}c@{}}4\\ 16\\ 32\end{tabular}}                        & \begin{tabular}[c]{@{}l@{}}12\\ 7\\ 8\end{tabular}   & \begin{tabular}[c]{@{}l@{}}10\\ 8\\ 8\end{tabular}   & \begin{tabular}[c]{@{}l@{}}11\\ 6\\ 8\end{tabular}   \\
                      & \multicolumn{1}{l|}{}                                                               &                                                      &                                                      &                                                      \\
        jython        & \textbf{\begin{tabular}[c]{@{}c@{}}4\\ 16\\ 32\end{tabular}}                        & \begin{tabular}[c]{@{}l@{}}13\\ 11\\ 9\end{tabular}  & \begin{tabular}[c]{@{}l@{}}14\\ 7\\ 8\end{tabular}   & \begin{tabular}[c]{@{}l@{}}13\\ 9\\ 8\end{tabular}   \\
                      & \multicolumn{1}{l|}{}                                                               &                                                      &                                                      &                                                      \\
        hsqldb        & \textbf{\begin{tabular}[c]{@{}c@{}}4\\ 16\\ 32\end{tabular}}                        & \begin{tabular}[c]{@{}l@{}}25\\ 17\\ 16\end{tabular} & \begin{tabular}[c]{@{}l@{}}22\\ 14\\ 18\end{tabular} & \begin{tabular}[c]{@{}l@{}}20\\ 16\\ 14\end{tabular} \\
                      & \multicolumn{1}{l|}{}                                                               &                                                      &                                                      &                                                      \\
        batik         & \textbf{\begin{tabular}[c]{@{}c@{}}4\\ 16\\ 32\end{tabular}}                        & \begin{tabular}[c]{@{}l@{}}23\\ 22\\ 21\end{tabular} & \begin{tabular}[c]{@{}l@{}}24\\ 20\\ 17\end{tabular} & \begin{tabular}[c]{@{}l@{}}25\\ 17\\ 17\end{tabular} \\ \bottomrule
        \end{tabular}
        \newline
        \caption[Multiple Classes Per Thread Time Examples]{Multiple Classes Per Thread Time Examples, with pool size: 4, 16, 32 and classes per thread: 2, 3, 4}
        \end{table}


        \begin{figure}[H]
            \centering
            \begin{tikzpicture}
\begin{axis}[ width=\textwidth, height=10cm, major x tick style = transparent, ybar=2*\pgflinewidth, bar width=14pt, ymajorgrids = true, ylabel = {time (sec)}, symbolic x coords={antlr, hsqldb, batik}, xtick = data, scaled y ticks = false, enlarge x limits=0.25, ymin=0, legend cell align=left, legend style={at={(1,1.05)}, anchor=south east, column sep=1ex} ]
    \addplot[style={rred,fill=rred,mark=none}]
        coordinates {(antlr, 18) (hsqldb, 22) (batik, 24)};
    \addplot[style={bblue,fill=bblue,mark=none}]
        coordinates {(antlr, 12) (hsqldb, 14) (batik, 20)};
    \addplot[style={ggreen,fill=ggreen,mark=none}]
        coordinates {(antlr, 13) (hsqldb,18) (batik, 17)};
    \legend{4 threads, 16 threads, 32 threads}
\end{axis}
            \end{tikzpicture}
            \caption{Increasing the number of threads with 3 classes per thread}
        \end{figure}
        \begin{figure}[H]
            \centering
            \begin{tikzpicture}
\begin{axis}[ width=\textwidth, height=10cm, major x tick style = transparent, ybar=2*\pgflinewidth, bar width=14pt, ymajorgrids = true, ylabel = {time (sec)}, symbolic x coords={antlr, hsqldb, batik}, xtick = data, scaled y ticks = false, enlarge x limits=0.25, ymin=0, legend cell align=left, legend style={at={(1,1.05)}, anchor=south east, column sep=1ex} ]
    \addplot[style={rred,fill=rred,mark=none}]
        coordinates {(antlr, 19) (hsqldb, 20) (batik, 25)};
    \addplot[style={bblue,fill=bblue,mark=none}]
        coordinates {(antlr, 14) (hsqldb, 16) (batik, 17)};
    \addplot[style={ggreen,fill=ggreen,mark=none}]
        coordinates {(antlr, 13) (hsqldb,14) (batik, 17)};
    \legend{4 threads, 16 threads, 32 threads}
\end{axis}
            \end{tikzpicture}
            \caption{Increasing the number of threads with 4 classes per thread}
        \end{figure}


        Based on the \texttt{Multiple Classes per thread} approach, our final version is presented below:
                \begin{figure}[H]
\begin{lstlisting}
public class Driver {
    public Driver(ThreadFactory factory, boolean ssa, int totalClasses) {
        _factory = factory;
        _ssa = ssa;
        _classCounter = 0;
        _sootClasses = new ArrayList<>();
        _totalClasses = totalClasses;
        _cores = Runtime.getRuntime().availableProcessors();
        _executor = new ThreadPoolExecutor(_cores/2, _cores, 0L, TimeUnit.MILLISECONDS, new LinkedBlockingQueue<Runnable>());
    }

    public void doInParallel(List<SootClass> sootClasses) {
        for(SootClass c : sootClasses)
            generate(c);
        _executor.shutdown();
        _executor.awaitTermination(Long.MAX_VALUE, TimeUnit.NANOSECONDS);
    }

    void generate(SootClass _sootClass) {
        _classCounter++;
        _sootClasses.add(_sootClass);
        if ((_classCounter % _classSplit == 0) || (_classCounter + _classSplit-1 >= _totalClasses)) {
            Runnable runnable = _factory.newRunnable(_sootClasses);
            _executor.execute(runnable);
            _sootClasses = new ArrayList<>();
        }
    }
}
\end{lstlisting}
        \caption{Final approach: Driver.java}
        \end{figure}

        \begin{figure}[H]
\begin{lstlisting}
public class ThreadFactory {
    public Runnable newRunnable(List<SootClass> sootClasses) {
        if (_makeClassGenerator)
            return new FactGenerator(_factWriter, _ssa, sootClasses);
        else
            return new FactPrinter(_ssa, _toStdout, _outputDir, _printWriter, sootClasses);
    }
}

public class FactGenerator implements Runnable {
    public void run() {
        for (SootClass _sootClass : _sootClasses) {
            /* for all soot classes generate like the sequential FactGenerator */
            /* ... */
        }
    }
}
\end{lstlisting}
        \caption{Final approach: ThreadFactory.java, FactGenerator.java}
        \end{figure}
        At first, the Main method calls \texttt{driver.doInParallel(classes)} which takes as an argument all the \texttt{SootClasses} to be generated. \texttt{Driver.java} in turn calculates the available threads to run the fact generation proccess and then calls \texttt{Driver.generate(c)} method for each SootClass. Then the \texttt{Driver.generate} creates a new \texttt{ThreadFactory} thread for each \_classSplit classes (e.g \_classSplit = 3 classes). \texttt{ThreadFactory} in turn calls the original \texttt{FactGenerator} (from the sequential approach) but instead of having a \texttt{generate} method, it has a \texttt{run} method with the same body.

\thesissection{Locking}
	Along with threads come locks. An incorrect use of locking could have very negative effects on the results. Having more locks than we actually needed would had lead to a time result as slow as the sequential one. Having less, would had lead to races and deadlocks. So, locks should be used very consciously.
    \\
    \\
	In the very beggining of this project, we tried a lot of different locking approaches such as lock everything. Of course the results were preety much similar as the sequential ones. So we had to do a more aggressive locking and also change some global objects to thread safe. We first tried to understand which part of Soot, Doop uses to generate the facts, and then tried to make that specific part thread safe.


    \thesissubsection{Doop Side}
    	In Doop the only two things we had to lock to prevent races and deadlocks, were the writting of the output file and three methods that were accessing a global field at sootMethod. We explain below in more detail.

    	\thesissubsubsection{CSVDatabase}
	    	In the file \texttt{CSVDatabase.java} we had to lock the output file to prevent more than one thread to write at the same time. By synchronizing the predicate file, we ensure that output files are written by one thread at a time, and also have the right data.
	    	\begin{figure}[H]
\begin{lstlisting}
synchronized(predicateFile) {
    Writer writer = getWriter(predicateFile);
    addColumn(writer, arg, shouldTruncate);
    for (Column col : args)
        addColumn(writer.append(SEP), col, shouldTruncate);
    writer.write(EOL);
}
\end{lstlisting}
	        \caption{CSVDatabase.java}
	        \end{figure}

	    \thesissubsubsection{Represantation}
	    	The other synchronization we had to provide was in three methods in the \texttt{Represantetion.java} file. Those methods were accessing and trying to retrieve some fileds from a global object at their SootMethod argument while threads were still active.
	    	\begin{figure}[H]
\begin{lstlisting}
public synchronized String signature(SootMethod) { /*...*/ }
public synchronized String handler(SootMethod, Trap, Session) { /*...*/ }
public synchronized String compactMethod(SootMethod) { /*...*/ }
\end{lstlisting}
	        \caption{Represantation.java}
	        \end{figure}



    \thesissubsection{Soot Side}
	    The way Soot is implemented, it has a class that encloses all global objects (\texttt{G.java}). As we mentioned before, Soot does much more than just translating bytecode to jimple, so it has various phases and a lot of different options for transformations given. The phase that translates bytecode to jimple is jb phase. In this phase we found all global objects, the methods that affects them and the methods that those global objects call and we synchronized them.

	    \thesissubsubsection{Type Assigner}
	    	The class \texttt{JimpleBodyPack} applies the transformations corresponding to the given options. In our case, it applies a \texttt{"jb.tr"} which means \textit{"jimple body transformation"}. From bytecode to jimple translation this is the only pack needed, so we lock before applying Type Assigner and unlock afterwards. All the transformations that take place to permform the bytecode to Jimple translation are first inserted and then retrieved from \texttt{packManager}, a global object. So to prevent race conditions, we changed the \texttt{packManager} from global to thread-local. The new \texttt{packManager} that we use is \texttt{PackManager}.
	    	\begin{figure}[H]
\begin{lstlisting}
lock.lock();
PackManager.v().getTransform("jb.tr").apply(b);
lock.unlock();
\end{lstlisting}
	        \caption{JimpleBodyPack.java}
	        \end{figure}

	    \thesissubsubsection{Pack Manager}
	    	As we mentioned before, the class that manages the Packs containing the various phases and their options is \texttt{PackManager}. We lock before retrieving Class Hierarchy Analysis and unlock afterwards.
			\begin{figure}[H]
\begin{lstlisting}
lock.lock();
p.add(new Transform("cg.cha", CHATransformer.v()));
p.add(new Transform("cg.spark", SparkTransformer.v()));
p.add(new Transform("cg.paddle", PaddleHook.v()));
lock.unlock();
\end{lstlisting}
	        \caption{PackManager.java}
	        \end{figure}

	    \thesissubsubsection{Shimple -ssa}
	    	We also use Soot to produce \textbf{S}h\textbf{imple} which is an \textbf{SSA} variation of Jimple. The way Shimple is generated is like Jimple but instead of producing Jimple in the final transformation, \texttt{Shimple.java} is called, which is the class that handles the translation from jimple to shimple. So we synchronized all Shimple-Body-creation methods that access global objects. To generate Shimple insted of Jimple, Soot must be given the \texttt{-ssa} flag.


\thesissection{Time Results}
	Summarizing, below are presented fact generation times for the previous version of Soot (2.5.0), the latest one and all our four approaches.

    \begin{figure}[H]
        \centering
        \begin{tikzpicture}
\begin{axis}[ width=\textwidth, height=12cm, major x tick style = transparent, ybar=2*\pgflinewidth, bar width=14pt, ymajorgrids = true, ylabel = {time (sec)}, symbolic x coords={antlr, hsqldb, batik}, xtick = data, scaled y ticks = false, enlarge x limits=0.25, ymin=0, legend cell align=left, legend style={at={(1,1.05)}, anchor=south east, column sep=1ex} ]
\addplot[style={mmaroon,fill=mmaroon,mark=none}]
    coordinates {(antlr, 75) (hsqldb, 83) (batik, 66)};
\addplot[style={rred,fill=rred,mark=none}]
    coordinates {(antlr, 49) (hsqldb, 52) (batik, 45)};
\addplot[style={oorange,fill=oorange,mark=none}]
    coordinates {(antlr, 22) (hsqldb,25) (batik, 22)};
\addplot[style={ggreen,fill=ggreen,mark=none}]
    coordinates {(antlr, 13) (hsqldb, 18) (batik, 15)};
\addplot[style={bblue,fill=bblue,mark=none}]
    coordinates {(antlr, 11) (hsqldb, 17) (batik, 13)};
\addplot[style={ccyan,fill=ccyan,mark=none}]
    coordinates {(antlr, 11) (hsqldb, 16) (batik, 12)};
\legend{Soot 2.5.0, Soot Latest Version, Fork-Join Framework, Thread/Methods, Thread/Class, Thread/Classes}
\end{axis}
        \end{tikzpicture}
        \caption{Fact Generation Time Results}
    \end{figure}

    \begin{table}[H]
    \centering
    \begin{tabular}{@{}l|lllll@{}}
    \toprule
    \textbf{Jars}     & \multicolumn{5}{c}{\textbf{Time (sec.)}}                                                                           \\ \midrule
    \textbf{Approach} & \textbf{Sequential} & \textbf{Fork/Join} & \textbf{Thread/Method} & \textbf{Thread/Class} & \textbf{Thread/Classes} \\ \midrule
    antlr             & 48                  & 23                 & 13                     & 13                    & 12                      \\
    eclipse           & 27                  & 13                 & 7                      & 7                     & 6                       \\
    jython            & 32                  & 16                 & 8                      & 8                     & 7                       \\
    hsqldb            & 50                  & 25                 & 15                     & 14                    & 14                      \\
    batik             & 63                  & 34                 & 18                     & 17                    & 17                      \\ \bottomrule
    \end{tabular}
    \newline
    \caption[Summarizing best times of all approaches]{Summarizing best times of all approaches with pool size 16-32}
    \end{table}

    In conclusion, just by changing the soot version and applying the latest one, gave a speedup up to 30-40\%. With the latest soot-version and our best approach we achived a speedup of 60-80\%. Our third attempt (Fork/Join Framework) was not a successful one, it was better than the sequential fact generation (obviously) but not as good as the other three.
    We used very few locks in both Doop and Soot so the locking didn't affect much our results. We also have to mention that with those locks Soot is thread safe for the way it is used by Doop (Java bytecode to Jimple or Shimple translation), not for all the other functionalities it has.




\begin{thesisabbreviations}[Acronyms and Abbreviations]
	\begin{tabularx}{\textwidth}{|X|X|}
        \hline
        IR & Intermediate Representation \\
        \hline
        SSA & Static Single Assignment \\
        \hline
        Jimple & Soot typed 3-address IR \\
        \hline
        Shimple & An SSA-version of Jimple \\
        \hline
        FG & Fact Generation \\
        \hline
        jb & Jimple Body \\
		\hline
	\end{tabularx}
\end{thesisabbreviations}


\begin{thesisbibliography}[References]{99}
	\bibitem{Doop: Framework for Java Pointer Analysis}
		"Doop: Framework for Java Pointer Analysis"

        [Online] \\ Available: \url{http://doop.program-analysis.org/} \\
	\bibitem{Sable: Soot}
		"Sable: Soot"

        [Online] \\ Available: \url{https://sable.github.io/soot/} \\
	\bibitem{Apache Ant}
		"Apache Ant"

        [Online] \\ Available: \url{http://ant.apache.org/} \\
    \bibitem{Oracle Java Fork/Join Framework}
        "Oracle Java Fork/Join Framework"

        [Online] \\ Available: \url{https://docs.oracle.com/javase/tutorial/essential/concurrency/forkjoin.html} \\
\end{thesisbibliography}


\end{document}
