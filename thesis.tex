%%
%% sample.tex - Documentation for dithesis Latex class
%% Copyright (C) 2011 Yannis Mantzouratos <mantzouratos@gmail.com>
%%
%% This file comes with Arial support.
%%
%% LICENSE:
%%
%% This work may be distributed and/or modified under the conditions of the
%% LaTeX Project Public License, either version 1.3 of this license or (at
%% your option) any later version.
%%
%% The latest version of this license is in:
%% http://www.latex-project.org/lppl.txt
%% and version 1.3 or later is part of all distributions of LaTeX version
%% 2005/12/01 or later.
%%
%% This work has the LPPL maintenance status "maintained".
%% The Current Maintainer of this work is Yannis Mantzouratos.
%%
%% This work consists of the source file dithesis.cls and the documentation
%% files sample.tex, samplewArial.tex, sample.pdf and samplewArial.pdf.
%% To ensure proper compilation, however, the logo of the University of Athens
%% is also distributed alongside this work, under the filename athena.jpg.
%%
%% NOTES and WARRANTY:
%%
%% This work conforms to the requirements of the Department of Informatics and
%% Telecommunications at the University of Athens regarding the preparation of
%% undergraduate theses, as of Sep 1, 2011.
%%
%% This work is distributed in the hope that it will be useful, but WITHOUT ANY
%% WARRANTY; without even the implied warranty of MERCHANTABILITY or FITNESS
%% FOR A PARTICULAR PURPOSE.
%% The entire risk as to the quality and performance of this work is with you.
%% Should this work prove defective, you assume the cost of all necessary
%% servicing, repair, or correction.
%% See the LaTeX Project Public License for more details.
%%
%% The latest official Microsoft Word(...) template can be found in
%% http://www.di.uoa.gr/lib.
%%

%% Compiled to PDF with MikTex (interpreter: XeLaTeX) on a Windows platform
\documentclass{dithesis}

\usepackage{mathspec}
\usepackage{xgreek}
\usepackage{xunicode}
\usepackage{xcolor}
\usepackage{pgfplots}
\usepackage{tikz}

\setallmainfonts[Mapping=tex-text]{Arial}
\setallsansfonts[Mapping=tex-text]{Arial}
\setallmonofonts[Mapping=tex-text]{Arial}

\renewcommand{\university}{Εθνικό και Καποδιστριακό Πανεπιστήμιο Αθηνών}
\renewcommand{\school}{Σχολή Θετικών Επιστημών}
\renewcommand{\department}{Τμήμα Πληροφορικής και Τηλεπικοινωνιών}

\renewcommand{\thesisplace}{Αθήνα}
\renewcommand{\thesisdate}{Σεπτέμβρης 2016}

\renewcommand{\thesislabel}{Πτυχιακή Εργασία}
\renewcommand{\supervisorlabel}{Επιβλέπων}
\renewcommand{\idlabel}{1115201200114}

\begin{document}
\thesistitle{Doop-Soot: Parallel Fact Generation}
\thesisauthor{Μούρης Δημήτριος}{1115201200114}
\supervisor{Σμαραγδάκης Γιάννης}{Καθηγητής ΕΚΠΑ}
\maketitle

\begin{thesisabstract}[Περίληψη]
    Παραλληλοποίηση του Fact Generation του Doop. Το Doop χρησιμοποιείται για μπλαμαπλπαλαμπλ

    \thesiskeywords{Θεματική Περιοχή}{Τεκμηρίωση}
                 {Λέξεις Κλειδιά}{Static Program Analysis}
                                 {Doop: Fact Generation}
                                 {Soot}
                                 {Πτυχιακές Εργασίες}
                                 {Τμήμα Πληροφορικής και Τηλεπικοινωνιών}
                                 {Πανεπιστήμιο Αθηνών}
\end{thesisabstract}

\begin{thesisabstract}[Abstract]
    In this paper, we provide documentation for the \LaTeX{} document class
    dithesis, which can be used for preparing undergraduate theses at the 
    Department of Informatics and Telecommunications, University of Athens.
    The class conforms to all requirements imposed by the Library, as of September
    2011.
    My thesis, which was based on the dithesis class, was accepted by the Library
    sometime during the summer semester of 2011.

    \thesiskeywords{Subject Area}{Documentation}
                                {Keywords}{Static Program Analysis}
                                {Doop: Fact Generation}
                                {Soot}
                                {Undergraduate Theses}
                                {Dept. of Informatics}
                                {University of Athens}
\end{thesisabstract}

\begin{thesisdedication}
Αφιέρωση σε κάποιους.
\end{thesisdedication}

\begin{thesisacknowledgments}[Ευχαριστίες]
    Ακολουθεί δείγμα ευχαριστιών.

    Θα ήθελα να ευχαριστήσω τον επιβλέποντα κ. Αλέξη Δελή για τη συνεργασία και τη
    βοήθεια κατά την εκπόνηση αυτής της πτυχιακής.

    Θα ήθελα επίσης να ευχαριστήσω το φίλο μου Μένιο για τις πολύτιμες
    παρατηρήσεις του σε προκαταρκτικές εκδόσεις του κειμένου.
\end{thesisacknowledgments}

\tableofcontents
\listoffigures
\listoftables

\begin{thesisprologue}[Πρόλογος]
    Το παρόν έγγραφο δημιουργήθηκε στην Αθήνα, το 2016, στα πλαίσια της 
    τεκμηρίωσης της κλάσσης \LaTeX{} dithesis.
    Η κλάσση αυτή διανέμεται με την ελπίδα ότι θα αποδειχθεί χρήσιμη, παρόλα αυτά 
    \emph{χωρίς καμιά εγγύηση}: χωρίς ούτε και την σιωπηρή εγγύηση 
    εμπορευσιμότητας ή καταλληλότητας για συγκεκριμένη χρήση.
    Για περισσότερες λεπτομέρειες, ανατρέξτε στην άδεια LaTeX Project Public 
    License.
\end{thesisprologue}

\thesissection{Εισαγωγή}
    eisagwgh gia doop kai soot

\thesissection{Doop}
    Doop is a framework for pointer, or points-to, analysis of Java programs. 
    Doop implements a range of algorithms, including context insensitive, call-site sensitive, 
    and object-sensitive analyses, all specified modularly as variations on a common code base.
    
    \thesissubsection{Fact Generation}
        Doop before running a pointer or points-to analysis, intergrates with Soot to generate
        the facts. Facts are in Jimple (Java sIMPLE), a typed 3-address IR suitable for performing
        optimizations, it only has 15 statements.

    \thesissubsection{Fact Generation Time Examples}
        \definecolor{bblue}{HTML}{4F81BD}
        \definecolor{rred}{HTML}{C0504D}
        \definecolor{ggreen}{HTML}{9BBB59}
        \definecolor{ppurple}{HTML}{9F4C7C}

        \begin{tikzpicture}
            \begin{axis}[width  = 0.85*\textwidth, height = 8cm, major x tick style = transparent,
                            ybar=2*\pgflinewidth, bar width=14pt, ymajorgrids = true, ylabel = {Run time speed},
                            symbolic x coords={antlr, hsqldb, batik}, xtick = data, scaled y ticks = false,
                            enlarge x limits=0.25, ymin=0, legend cell align=left, 
                            legend style={at={(1,1.05)}, anchor=south east, column sep=1ex}
                        ]
                \addplot[style={bblue,fill=bblue,mark=none}]
                    coordinates {(antlr, 75) (hsqldb, 83) (batik, 66)};

                \addplot[style={rred,fill=rred,mark=none}]
                     coordinates {(antlr, 49) (hsqldb, 52) (batik, 45)};

                \addplot[style={ggreen,fill=ggreen,mark=none}]
                     coordinates {(antlr, 22) (hsqldb,25) (batik, 22)};

                \addplot[style={ppurple,fill=ppurple,mark=none}]
                     coordinates {(antlr, 12) (hsqldb, 17) (batik, 14)};

                \legend{Doop-Nexgen, Soot Latest Ver., Fork-Join Framework, Threads/Methods}
            \end{axis}
        \end{tikzpicture}


\thesissection{Soot}
    Originally, Soot started off as a Java optimization framework. By now, researchers and 
    practitioners from around the world use Soot to analyze, instrument, optimize and 
    visualize Java and Android applications.





\end{document}
